\documentclass[11pt,a4paper,twoside,openany]{book}
\usepackage[utf8]{inputenc}
\usepackage[T1]{fontenc}
\usepackage{fullpage}

\begin{document}
\chapter*{Descobrindo Inteligência Artificial}

Pense na forma como você aprendeu durante a sua vida até aqui. Em algum momento, você teve que combinar informações não-correlatas para poder chegar a uma conclusão inteligente sobre um assunto que aprendia ou solucionar um problema.

Ao longo do tempo, cientistas e programadores vêm tentando imitar essa abordagem humana para o aprendizado e para a solução de problemas. As duas áreas da ciência, assim como no homem, são bastante próximas mas não são a mesma coisa. Uma coisa é aprender algo novo, memorizar informação e recuperá-la com base em um contexto ou reagindo a uma pergunta. Outra coisa bastante diferente é solucionar um problema utilizando informações que você adquiriu após algum tempo de estudo. 

Para usar uma analogia simples, todo mundo aprendeu no ensino médio a equação que relaciona os tamanhos dos lados do triângulo retângulo: a soma dos quadrados dos lados é igual ao quadrado da hipotenusa (você provavelmente aprendeu isso como "Teorema de Pitágoras"). Uma coisa é decorar a fórmula e operá-la, substituindo valores e chegando a resultados. Outra coisa totalmente diferente é deduzir a equação, prová-la verdadeira e decidir quando aplicá-la num problema de cunho prático.

Ou seja, inteligência é a tomada de decisão, é a geração de novos conhecimentos a partir de conhecimentos adquiridos previamente.

No mundo tecnológico de hoje temos duas abordagens principais: aprendizado de máquina, que modela o aprendizado de uma entidade artificial segundo regras específicas e eventual validação humana; e inteligência artificial que permite à máquina tomar decisões baseado no que percebe do mundo à sua volta.

Note que a percepção de uma máquina (exatamente como acontece com humanos, inclusive) está limitada a quantidade de sensores que ela possui. Uma máquina, por exemplo, tem sérias dificuldades em inferir sobre a cor de um objeto se a única ferramenta sensorial que ela possui é um sonar (da mesma forma que um homem com deficiência visual não poderia inferir sobre a cor desse mesmo objeto se não tivesse outras indicações, como textura ou o fato de que aquele objeto é fabricado sempre com uma cor só, por exemplo).

O objetivo de uma máquina dotada de inteligência artificial é sempre o de maximizar o resultado obtido dado um conjunto de restrições. Pense novamente no teorema de pitágoras. Existem diversas outras formas de indicar a relação existente entre o comprimento da hipotenusa e dos lados, de métodos geométricos a identidades. O teorema criado por Pitágoras, contudo, ficou famoso por ser o mais simples e didático criado até agora. Se a uma máquina fosse dada a tarefa de equacionar a relação dos comprimentos dos lados de um triângul  retângulo, ela chegaria a mesma solução? Assumindo que o Teorema de Pitágoras é a solução ótima (a melhor forma de representar e perpetuar o conhecimento) e que a máquina teria tempo suficiente para isso, talvez chegasse (o resultado depende das restrições bem definidas e do tempo necessário, bem como do método aplicado para a resolução do problema).

Uma máquina comum (o seu computador ou smartphone, por exemplo) tomam decisões o tempo todo. Qual janela exibir? Qual cor mostrar nessa tela ou nesse contexto? Em um segundo, seu computador provavelmente toma 1 bilhão de decisões. Então o seu computador é uma máquina de inteligência artificial? Sim e não.

Sim, porque em teoria, ele poderia ser. Não, porque na prática ele toma decisões sobre um domínio restrito de regras (aquelas programadas pelo desenvolvedor) que foram previamente estabelecidas. Um computador sabe decidir entre qual é o maior entre dois números, mas não sabe decidir qual é o maior entre dois elefantes. Ainda que você e eu tenhamos uma noção intrínseca de que se trata da mesma habilidade.

Como, então, transformar o seu computador numa máquina inteligente? É isso que discutiremos nos próximos capítulos. Falaremos das diferentes abordagens da inteligência artificial, suas aplicações, seus paradigmas e seus problemas atuais.
\end{document}
